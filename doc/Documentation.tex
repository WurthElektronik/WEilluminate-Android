
\documentclass[envcountsame,envcountchap, deutsch]{i-studis}

\usepackage{makeidx}         	% Index
\usepackage{multicol}        	% Zweispaltiger Index
\usepackage{float}				% Platzierung von tables (H)
%\usepackage[bottom]{footmisc}	% Erzeugung von Fu�noten

%%-----------------------------------------------------
%\newif\ifpdf
%\ifx\pdfoutput\undefined
%\pdffalse
%\else
%\pdfoutput=1
%\pdftrue
%\fi
%%--------------------------------------------------------
%\ifpdf
\usepackage[pdftex]{graphicx}
\usepackage{epstopdf}
\usepackage[pdftex,plainpages=false]{hyperref}
%\else
%\usepackage{graphicx}
%\usepackage[plainpages=false]{hyperref}
%\fi

%%-----------------------------------------------------
\usepackage{color}				% Farbverwaltung
\usepackage[english]{babel}
\usepackage[utf8]{inputenc}  	% Erm�glicht Umlaute-Darstellung unter Linux (je nach verwendetem Format)

%-----------------------------------------------------
\usepackage{listings} 			% Code-Darstellung
\lstset
{
	basicstyle=\scriptsize, 	% print whole listing small
	keywordstyle=\color{blue}\bfseries,
								% underlined bold black keywords
	identifierstyle=, 			% nothing happens
	commentstyle=\color{red}, 	% white comments
	stringstyle=\ttfamily, 		% typewriter type for strings
	showstringspaces=false, 	% no special string spaces
	framexleftmargin=7mm, 
	tabsize=3,
	showtabs=false,
	frame=single, 
	rulesepcolor=\color{blue},
	numbers=left,
	linewidth=146mm,
	xleftmargin=8mm
}

\usepackage{textcomp} 			% Celsius-Darstellung
\usepackage{amssymb,amsfonts,amstext,amsmath}	% Mathematische Symbole
\usepackage[german, ruled, vlined]{algorithm2e}
\usepackage[a4paper]{geometry} % Andere Formatierung
\usepackage{bibgerm}
\usepackage{array}
\usepackage{hyperref}
\hyphenation{Ele-men-tar-ob-jek-te  ab-ge-tas-tet Aus-wer-tung House-holder-Matrix Le-ast-Squa-res-Al-go-ri-th-men} 		% Weitere Silbentrennung bei Bedarf angeben
\setlength{\textheight}{1.1\textheight}
\pagestyle{myheadings} 			% Erzeugt selbstdefinierte Kopfzeile
\makeindex 						% Index-Erstellung
\expandafter\def\expandafter\UrlBreaks\expandafter{\UrlBreaks%  save the current one
	\do\a\do\b\do\c\do\d\do\e\do\f\do\g\do\h\do\i\do\j%
	\do\k\do\l\do\m\do\n\do\o\do\p\do\q\do\r\do\s\do\t%
	\do\u\do\v\do\w\do\x\do\y\do\z\do\A\do\B\do\C\do\D%
	\do\E\do\F\do\G\do\H\do\I\do\J\do\K\do\L\do\M\do\N%
	\do\O\do\P\do\Q\do\R\do\S\do\T\do\U\do\V\do\W\do\X%
	\do\Y\do\Z} % Dient der Darstellung der URL's im Literaturverzeichnis (Zeilenumbrüche)

%--------------------------------------------------------------------------
\begin{document}
%------------------------- Titelblatt -------------------------------------

%--------------------------------------------------------------------------
\frontmatter 
%--------------------------------------------------------------------------
\chapter*{Introduction}

The two Android applications \glqq WEilluminate\grqq\ and \glqq Proteus Connect\grqq, collectively named \glqq App\grqq\ from here on, are Apps provided by Würth Elektronik eiSos GmbH \& Co. KG for communication over the Bluetooth Low Energy standard with modules of the Proteus Series.

The Apps access the Bluetooth functionality of the users smartphone. No internet connection is required. The \textbf{minimum required Android API is 21} (Android 5.1). Also the smartphone has to support a MTU (Maximum Transfer Unit) higher than 21. Is this not the case, the app can't be used on the device.
\tableofcontents 						% Inhaltsverzeichnis
%--------------------------------------------------------------------------
\mainmatter                        		% Hauptteil (ab hier arab. Seitenzahlen)
%--------------------------------------------------------------------------
% Die Kapitel werden in separaten .tex-Dateien abgelegt und hier eingebunden.
\chapter{Project Structure}

The project structure of the apps is nearly identical. The base identifiers of the apps are:

\begin{itemize}
	\item[1.] com.eisos.android (WEilluminate)
	\item[2.] com.eisos.android.terminal (Proteus Connect)
\end{itemize}

\noindent
The applications contain following packages:

\begin{itemize}
	\item adapter
	\item bluetooth
	\item customLayout
	\item database
	\item frags
	\item profiles
	\item utils
\end{itemize}

\noindent
The Proteus Connect app additionally has the \glqq gpio\grqq\ package because this app can control GPIO pins whereas the WEillumiante app cannot.

\section{Packages}

\subsection{adapter}

In this package all the adapters of the List- and RecyclerViews are saved. They are needed to give list entries a custom look.

\subsection{bluetooth}

This is one of the core packages of the app. Most of the classes have been created with the help of the Nordic Semiconductor Bluetooth libraries like the Android-BLE-Library (\url{https://github.com/NordicSemiconductor/Android-BLE-Library}) and the Android-Scanner-Compat-Library (\url{https://github.com/NordicSemiconductor/Android-Scanner-Compat-Library}). Everything that is related to Bluetooth and the connection in general can be found here. The most important classes are \textit{UARTManager}, \textit{BleMulticonnectProfileService} and \textit{UARTService}.

\subsection{customLayout}

In this package classes are saved, that specifically are used to expand Layout classes.

\subsection{database}

Everything related to databases, that means saving favorite devices or profiles, can be found in this package.

\subsection{frags}

All the necessary Fragment classes of the UI are saved here.

\subsection{gpio (Proteus Connect only)}

Everything related to GPIO is saved in here.

\subsection{profiles}

All classes that are needed to create, edit or delete profiles are located in this package.

\subsection{utils}

In here, helper classes like conversion classes etc. are saved.\\

\noindent
Most of the Activity-Classes are located directly in the identifier package.

\section{Resources}

In Android there are multiple resource folders. They can be found under \textit{res}.

\begin{itemize}
	\item drawable (Images and icons)
	\item layout (UI design, layout xml-files)
	\item menu (All menu option xml-files)
	\item mipmap (The launcher icons)
	\item values (App colors and string resources)
	\item xml (Contains file path xml-files)
\end{itemize}

\section{Global settings}

In an Android project there are two main files for global app settings. The first one is the \textit{Manifest.xml}. In this file all the Activities, Services and Permissions are declared. The second one is the \textit{build.gradle(:app)}. All needed project dependencies and the app versions are set in this file.




% ...
%--------------------------------------------------------------------------
\backmatter                        		% Anhang
%--------------------------------------------------------------------------
%--------------------------------------------------------------------------
\printindex 							% Index (optional)
%--------------------------------------------------------------------------

\end{document}
