\chapter{Project Structure}

The project structure of the apps is nearly identical. The base identifiers of the apps are:

\begin{itemize}
	\item[1.] com.eisos.android (WEilluminate)
	\item[2.] com.eisos.android.terminal (Proteus Connect)
\end{itemize}

\noindent
The applications contain following packages:

\begin{itemize}
	\item adapter
	\item bluetooth
	\item customLayout
	\item database
	\item frags
	\item profiles
	\item utils
\end{itemize}

\noindent
The Proteus Connect app additionally has the \glqq gpio\grqq\ package because this app can control GPIO pins whereas the WEillumiante app cannot.

\section{Packages}

\subsection{adapter}

In this package all the adapters of the List- and RecyclerViews are saved. They are needed to give list entries a custom look.

\subsection{bluetooth}

This is one of the core packages of the app. Most of the classes have been created with the help of the Nordic Semiconductor Bluetooth libraries like the Android-BLE-Library (\url{https://github.com/NordicSemiconductor/Android-BLE-Library}) and the Android-Scanner-Compat-Library (\url{https://github.com/NordicSemiconductor/Android-Scanner-Compat-Library}). Everything that is related to Bluetooth and the connection in general can be found here. The most important classes are \textit{UARTManager}, \textit{BleMulticonnectProfileService} and \textit{UARTService}.

\subsection{customLayout}

In this package classes are saved, that specifically are used to expand Layout classes.

\subsection{database}

Everything related to databases, that means saving favorite devices or profiles, can be found in this package.

\subsection{frags}

All the necessary Fragment classes of the UI are saved here.

\subsection{gpio (Proteus Connect only)}

Everything related to GPIO is saved in here.

\subsection{profiles}

All classes that are needed to create, edit or delete profiles are located in this package.

\subsection{utils}

In here, helper classes like conversion classes etc. are saved.\\

\noindent
Most of the Activity-Classes are located directly in the identifier package.

\section{Resources}

In Android there are multiple resource folders. They can be found under \textit{res}.

\begin{itemize}
	\item drawable (Images and icons)
	\item layout (UI design, layout xml-files)
	\item menu (All menu option xml-files)
	\item mipmap (The launcher icons)
	\item values (App colors and string resources)
	\item xml (Contains file path xml-files)
\end{itemize}

\section{Global settings}

In an Android project there are two main files for global app settings. The first one is the \textit{Manifest.xml}. In this file all the Activities, Services and Permissions are declared. The second one is the \textit{build.gradle(:app)}. All needed project dependencies and the app versions are set in this file.



